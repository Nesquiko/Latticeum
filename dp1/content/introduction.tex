\chapter{Introduction}

Ethereum, a decentralized blockchain platform launched in 2015, provides a
Turing-complete execution environment for smart contracts and decentralized
applications. Its modular architecture comprises a proof-of-stake consensus
layer (Gasper) and an execution layer governed by the Ethereum Virtual Machine
(EVM). A primary challenge for Ethereum is scalability, as the need for all
nodes to validate and re-execute every transaction limits throughput and can
centralize the network by increasing resource demands.

"Snarkification," the integration of Zero-Knowledge Proofs (ZKPs) like SNARKs,
offers a path to mitigate these scalability issues. Initiatives such as the
'Beam Chain' aim to snarkify consensus, allowing validators to verify compact
proofs instead of entire state components. Similarly, Zero-Knowledge Ethereum
Virtual Machines (ZkEVMs) seek to make EVM execution provable, enabling nodes
to verify block execution via a single proof. Enshrining a ZkEVM into
Ethereum's Layer 1 could further enhance scalability and simplify the creation
of native rollups.

Zero-Knowledge proofs allow proving a statement's truth without revealing
underlying information. The statement are usually a membership in NP language,
thus an entire computations can be proven with this primitives, for example
like computations done by the EVM, turning it into a ZkEVM. However,
developing ZkEVMs is challenging due to the EVM's original design: complex
opcodes, large word sizes requiring range proofs, its stack-based nature, and
ZK-unfriendly storage mechanisms. Early solutions utilized recursive proofs to
aggregate computation steps, leading to the first generation of Layer 2
ZkEVMs. Folding schemes further optimized this by compressing instances. The
current trend involves Zero-Knowledge Virtual Machines (ZkVMs), especially
those based on the simpler, register-based RISC-V ISA, which is more amenable
to ZK proof generation.

While many ZKP systems rely on assumptions vulnerable to quantum attacks,
lattice-based cryptography provides a foundation for post-quantum secure ZK
solutions, with security often based on hard problems like SVP or SIS.

This thesis investigates the design and implementation of a RISC-V ZkVM using
lattice-based cryptography. A key application is proving Ethereum block
execution, contributing to the network's scalability and post-quantum
security. This work explores the practical construction of such a system and
aims to assess whether lattice-based cryptography can offer performance
comparable to classical ZKP methods while ensuring quantum resistance.

\section*{Document Structure}

This progress report on the DP1 solution details the initial analytical work
in Analysis \ref{chap:analysis}. This chapter begins with an overview of
Ethereum \ref{sec:ethereum}, covering its consensus and execution mechanisms.
It then explores the concept of Snarkification of Ethereum \ref{sec:snarkification},
discussing Zero-Knowledge Proofs and their application to both the consensus
layer via the Beam Chain initiative and the execution layer through enshrined
ZkEVMs. Following this, the report analyzes ZkEVMs \ref{sec:zkevm},
examining the challenges of proving EVM execution, the evolution of solutions
from recursive proofs and folding schemes to modern ZkVMs. Finally, the
analysis introduces Lattice-based cryptography \ref{sec:lattices}, outlining
Ajtai commitments and the potential of lattices for post-quantum
zero-knowledge systems, including recent developments like LaBRADOR,
Greyhound, and LatticeFold/LatticeFold+.

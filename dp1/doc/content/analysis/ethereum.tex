\section{Ethereum}\label{sub:ethereum}

Ethereum is a decentralized, open-source blockchain with smart contract
functionality \cite{ethereumEthereumWhitepaper}. Conceptualized by Vitalik
Buterin in 2013/2014, it went live in 2015, aiming to build a platform for
decentralized applications that adds Turing complete execution environment
to blockchain \cite{ethereumWhitePaperPdf}. Ethereum enables developers to
create and deploy decentralized applications and crypto assets.
The Ethereum network is comprised of two layers, two peer-to-peer networks,
consensus layer and execution layer.

\subsection{Consensus}\label{subsec:ethereum_consensus}
% very good explainer https://ethos.dev/beacon-chain

Consensus on Ethereum is achieved with proof-of-stake (PoS) consensus mechanism
called Gasper \cite{VitalikGasper}. PoS system relies on crypto-economic incentives,
rewarding honest stakers (people who put economic capital in the network)
and penalizing malicious ones.

Stakers, or also called validators, propose new blocks. Validator is selected for
a block proposal pseudo-randomly from the pool in each slot. A slot is some
time amount (as of writing of this work it is 12 seconds on Ethereum mainnet),
in which the pseudo-randomly chosen validator can propose new block. The software
creating the new block is called consensus client. Proposer's consensus client
requests a bundle of transactions from the execution layer \ref{subsec:ethereum_execution},
wraps them into a block and gossips (sends) the new block to other participants
over the consensus p2p network. The rest of the validator pool can in 32 slot
(or one epoch) attest to that new block's validity. In order for a block to be
finalized, it must be attested by a super-majority, which is 66\% of the total
balance of all validators. \cite{ethereumConsensusMechanisms}.

\subsection{Execution}\label{subsec:ethereum_execution}

Software operating on the execution layer is called execution client. Nodes on
execution layer hold the latest state and database of all Ethereum data. These
clients gossip incoming transactions over the execution layer p2p network, and
each stores them in a local mempool. Once they are put inside a block, each
transaction is executed in Ethereum-Virtual-Machine (EVM). EVM is a stack based
virtual machine operating 256 bit words which executes smart contract code.
Ethereum is a decentralized state machine, with rules defined by the EVM.
EVM can be thought of as a function. Given a valid state and valid set of
transactions\footnote{validity of transactions, and thus transitively validity
of state, is guaranteed by the consensus layer \ref{subsec:ethereum_consensus}},
outputs next valid state:

\[
	F(S_n, T) = S_{n+1} \label{eq:state_transition}
\]

Thus, Ethereum's state transition function is described by the EVM. This function
must be executed by each execution node for each new block in order to keep up
with the current chain state. \cite{ethereumEthereumVirtual}



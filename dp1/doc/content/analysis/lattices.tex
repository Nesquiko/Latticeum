\section{Lattice-based cryptography}\label{sec:lattices}

Lattice-based cryptography has emerged as a significant area of research,
particularly due to its potential for constructing cryptographic primitives
resistant to attacks by quantum computers. Unlike traditional public-key
cryptosystems such as those based on elliptic curves, which are vulnerable to
Shor's algorithm on a quantum computer, many lattice-based constructions are
believed to maintain their security in a post-quantum world. The security of
these schemes often relies on the presumed hardness of certain computational
problems on lattices, such as the Shortest Vector Problem (SVP) \cite{AjtaiLattices, LatticeTutorial}.

A lattice $L$ is a discrete subgroup of $\mathbb{R}^n$, typically
represented as the set of all integer linear combinations of a set of linearly
independent basis vectors $B = \{\mathbf{b}_1, \mathbf{b}_2, \dots, \mathbf{b}_m\}$
where $\mathbf{b}_i \in \mathbb{R}^n$. That is:

\[
	L(B) = \left\{ \sum_{i=1}^{m} x_i \mathbf{b}_i \mid x_i \in \mathbb{Z} \right\}
\]

The dimension of the lattice is $m$. Many cryptographic constructions are
built upon the difficulty of solving problems like finding the shortest
non-zero vector in a lattice (SVP) or finding the lattice point closest to a
given target vector (CVP), especially in high dimensions.

One of the pioneering works in this field was by Miklós Ajtai in 1996, who
introduced a cryptographic construction whose security could be based on the
worst-case hardness of lattice problems \cite{AjtaiLattices}. This provided a
stronger theoretical foundation for cryptosystems using lattices.

Lattice-based cryptography offers several advantages:

\begin{itemize}
    \item \textbf{Post-quantum security:} As mentioned, this is a primary driver for its adoption.
    \item \textbf{Efficiency:} Some lattice-based operations can be
		computationally efficient, particularly those involving matrix-vector
		multiplications over small integers \cite{LatticeTutorial}.
    \item \textbf{Versatility:} Lattices have been used to construct a wide
		array of cryptographic primitives, including public-key encryption,
		digital signatures, and importantly for this thesis, commitment
		schemes \cite{LatticeTutorial}.
\end{itemize}

\subsection{Ajtai Commitments}

A commitment scheme is a cryptographic primitive that allows a party (the
prover) to commit to a value while keeping it hidden from another party (the
verifier), with the ability to reveal the committed value later. The scheme
must satisfy two main properties:
\begin{itemize}
    \item \textbf{Hiding:} The verifier cannot learn the committed value from
		the commitment itself before the reveal phase.
    \item \textbf{Binding:} The prover cannot change the committed value after
		the commitment phase.
\end{itemize}

The Ajtai commitment scheme from Ajtai's 1996 work \cite{AjtaiLattices}, is a
lattice-based commitment scheme. Its security is typically based on the
hardness of the Shortest Integer Solution (SIS) problem \cite{AjtaiLattices, LatticeTutorial}.

The SIS problem is defined as follows: Given a random matrix
$A \in \mathbb{Z}_q^{n \times m}$ (where $q$ is a modulus, $n$ is the row
dimension, and $m$ is the column dimension, with $m > n \log q$), find a
non-zero integer vector $\mathbf{s} \in \mathbb{Z}^m$ with small norm (i.e.,
$s_i \in s; ||s_i|| \leq \beta$, for some bound $\beta$) such that:

\[
	A\mathbf{s} = \mathbf{0} \pmod{q}
\]

Finding such a short vector $\mathbf{s}$ is believed to be computationally
hard for appropriate parameters.

\begin{itemize}
    \item \textbf{Binding:} The binding property relies on the SIS assumption.
		If a prover could find two different openings (e.g., $\mathbf{s}$ and
		$\mathbf{s}'$ where $\mathbf{s} \neq \mathbf{s}'$) for the same
		commitment $\mathbf{t}$, such that $A\mathbf{s} = \mathbf{t} \pmod{q}$
		and $A\mathbf{s}' = \mathbf{t} \pmod{q}$, then $A(\mathbf{s} -
		\mathbf{s}') = \mathbf{0} \pmod{q}$. If $\mathbf{s} - \mathbf{s}'$ is
		a short non-zero vector, this would imply a solution to the SIS
		problem for matrix $A$.
    \item \textbf{Hiding:} The hiding property is based on the Learning With
		Errors (MLWE) problem or related assumptions. Informally, given $A$ and
		$A\mathbf{s} \pmod{q}$ where $\mathbf{s}$ is short and random, it is
		computationally difficult to recover $\mathbf{s}$. The distribution of
		$A\mathbf{s} \pmod{q}$ is computationally indistinguishable from a
		uniform random vector over $\mathbb{Z}_q^n$ \cite{LatticesInZKP}.
\end{itemize}

A useful feature of some Ajtai commitments is that the size of the
commitment $\mathbf{t}$ does not grow with the dimension of the
message $\mathbf{s}$, though the message components themselves must be
small \cite{LatticesInZKP}.

\subsection{Lattices in ZK}

Lattice-based cryptography presents a compelling avenue for advancing
zero-knowledge proof systems, due to its strong security foundations
against quantum computers. This makes them a promising candidate for
future-proofing ZK solutions.

\subsubsection{LaBRADOR: Compact Proofs for R1CS from Module-SIS}

LaBRADOR, introduced by Beullens and Seiler \cite{LaBRADOR}, is a lattice-based
proof system designed for Rank-1 Constraint Systems (R1CS). An advantage of
LaBRADOR is its ability to generate very compact proof sizes, particularly for
large circuits, making it more efficient in terms of proof transmission and
storage compared to many other post-quantum approaches. For instance, LaBRADOR
can prove knowledge of a solution for an R1CS modulo $2^{64} + 1$ with
$2^{20}$ constraints with a proof size of only 58 KB.

However, a notable drawback of LaBRADOR is that its verifier runtime is linear
in the size of the R1CS instance. This means that while the proofs are small,
the verification process is not succinct, which can limit its applicability in
scenarios requiring fast, sublinear verification. Despite this, LaBRADOR's
compact proof generation makes it a candidate as an "outer layer" or wrapper.
It can be used to prove the correctness of a (potentially much larger) proof
generated by another succinct, but perhaps less compact, proof system, thereby
achieving overall succinctness with post-quantum security.

\subsubsection{Greyhound: Fast Polynomial Commitments from Lattices}

Building upon the strengths of LaBRADOR, Nguyen and Seiler proposed Greyhound,
an efficient polynomial commitment scheme (PCS) based on standard
lattice assumptions \cite{Greyhound}. Greyhound leverages LaBRADOR as a core
component to achieve succinct proofs of polynomial evaluation.

The construction involves a three-round protocol for proving evaluations for
polynomials of bounded degree $N$ with a verifier time complexity of
$O_p(\sqrt{N})$. By composing this with the LaBRADOR proof system, Greyhound
obtains a succinct proof of polynomial evaluation that also has a sublinear
verifier runtime. For large polynomials (e.g., degree up to $N \approx
2^{30}$), Greyhound produces evaluation proofs of approximately 53KB. This
demonstrates the practical utility of using LaBRADOR to compress proofs for
more complex cryptographic primitives like polynomial commitments.

\subsubsection{LatticeFold and LatticeFold+}

In the domain of folding schemes, which are techniques for building efficient
recursive SNARKs, Boneh and Chen introduced LatticeFold. LatticeFold addressed
the challenge of maintaining low-norm witnesses through multiple rounds of
folding by employing a sumcheck-based range proof to ensure extracted
witnesses remained valid \cite{LatticeFold}.

More recently, Boneh and Chen proposed LatticeFold+
\cite{LatticeFoldPlus}, an improved lattice-based folding protocol.
LatticeFold+ enhances its predecessor in several key aspects: the prover is
substantially faster (five to ten times), the verification circuit is simpler,
and the folding proofs are shorter. These improvements are achieved through
two main lattice techniques:

\begin{enumerate}
    \item A new, purely algebraic range proof that is more efficient than the
		bit-decomposition approach used in LatticeFold.
    \item The use of double commitments (commitments of commitments) to
		further shrink proof sizes, along with a new sumcheck-based
		transformation for folding statements about these double commitments.
\end{enumerate}

LatticeFold+ aims to be competitive with, or even outperform, pre-quantum
folding schemes like HyperNova \cite{HyperNova} while offering plausible post-quantum security.

\section{Snarkification of Ethereum}\label{sec:snarkification}

As discussed in sections on consensus (\ref{subsec:ethereum_consensus}) and
execution layer (\ref{subsec:ethereum_execution}), nodes on each layer must expend
compute resources to validate the consensus data of new blocks or re-execute all
transactions within them to verify and maintain the blockchain's state.
These computational demands limit the scalability of the entire network.

Ethereum's decentralized node network includes not only participants with
capable servers, but also hobbyists, who use home internet connections and
less powerful machines. Naively attempting to scale the network, for instance,
by decreasing slot time, or increasing block size may raise bandwidth and node
requirements, thereby ousting less capable nodes and thus hindering Ethereum's
decentralization.

Snarkification refers to the process of integrating specific type of Zero
Knowledge Proofs (ZKPs) called SNARK (Succinct Non-interactive Argument of
Knowledge) in order to offload a computation to one entity (one node, or a
cluster that is a small subset of the whole network), which performs the
computation and generates a proof of its validity. This proof can then be
verified by the rest of the network at fractional compute cost.

\subsection{Zero Knowledge Proofs}

Zero-Knowledge Proofs are cryptographic primitives that allow a prover
to convince a verifier of a statement's truth without revealing any
information beyond the statement's validity itself. The statement typically
concerns membership in an NP language/relation \cite{GoldreichNPProofs}.
Traditionally, ZKPs are interactive protocols where a prover, knows a
private witness for a statement (e.g., a solution to a problem), aims to
convince a verifier of this knowledge. Through a series of back-and-forth
interactions, the verifier becomes convinced of the prover's claim (if true)
without learning the witness itself.

Interactive proofs where the verifier's messages consist only of random
challenges are known as public-coin protocols. The Fiat-Shamir heuristic \cite{FiatShamir}
can transform such public-coin interactive proofs into non-interactive
proofs by replacing the verifier's random challenges with challenges derived
from a cryptographic hash function applied to the protocol's transcript. This
non-interactivity allows a single generated proof to be validated by any
number of verifiers.

A SNARK is a specific type of ZKP. The 'Succinct' aspect implies that proof
sizes are very small (e.g., polylogarithmic in the size of the witness or
statement) and verification is faster than re-executing the original
computation, often polylogarithmic in the computation's complexity \cite{BCCT11}.

\subsection{Snarkifying consensus}

The 'Beam Chain' is a research initiative, introduced by Justin Drake at Devcon
2024 \cite{JustinDrakeBeamchain}, aimed at redesigning and improving
Ethereum's consensus mechanism. A key aspect of this proposal involves
leveraging SNARKs to make consensus layer state transitions provable, in near
real-time. Instead of each validator independently re-validating all
components of a new consensus state, they would verify a single, compact
SNARK. Based on this proof's validity, they would accept or
reject the proposed state update.\footnote{The entity proposing the state
update, such as a block proposer, would be responsible for generating and
submitting this proof.}

\subsection{Snarkifying execution}

Ethereum's execution state transitions are dictated by the rules of the
EVM. To reduce the computational burden of verifying execution using ZKPs, the
EVM's execution logic needs to be provable with a ZKPs, creating a ZkEVM. This
would change the state transition function (\ref{eq:state_transition}) to:

\[
	F(S_n, T) = (S_{n+1}, \pi)
\]

Where $\pi$ is the proof of the state transition's validity. Therefore,
verifying a block's execution state transition would involve verifying this
single proof, rather than re-executing all transactions within the block.
Furthermore, an increase in block size would have a little to no impact on
verification times for nodes, as proof verification complexity typically grows
much slower than the computation being proven. This could allow for larger
block sizes without risking the exclusion of nodes with modest computational
capabilities due to their weaker computational capabilities, thereby
supporting scalability.

Another approach is to 'enshrine' a ZkEVM directly into the Ethereum protocol (L1).
This offers similar benefits in terms of L1 state transition verification, as
the native EVM could delegate execution proving to this enshrined ZkEVM. In
addition, an enshrined ZkEVM could support and simplify the creation of what
are sometimes termed 'native rollups' \cite{NativeRollups}. These would be
Layer 2 solutions that could leverage the L1's enshrined ZkEVM for verifying
their own state transitions, which are themselves batches of EVM-compatible
transactions. The correctness of these L2 transaction batches would thus be
directly enforced by the L1 execution layer through ZK proof verification.
This could lead to faster, settlement for these L2s on L1, which in turn could
simplify composability between L1 and these native L2s, as well as among
different native L2s.


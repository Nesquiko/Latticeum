%	2. snarkification
%		a. ZKPs
%		b. snarkification benefits, why and how
%		c. beam chain - snarkification of consensus
%		d. enshrined zkEVM - snarkification of execution
\chapter{Snarkification of Ethereum}\label{chapter:snarkification}

As discussed in sections about consensus (\ref{sec:ethereum_consensus}) and
execution layer (\ref{sec:ethereum_execution}), nodes on each layer must spend
compute resources to validate the consensus data of new blocks or re-execute all
transactions within them to maintain the blockchain's state.
These computations limit the scalability of the whole network.

Ethereum's strong
decentralized node set includes not only participants with capable servers,
but also hobbyists, who use home internet connections and low-end machines.
Naively scaling the network, for instance, by decreasing slot time, or increasing
block size may raise bandwidth and node requirements, which would exclude
less capable nodes and thus hinder Ethereum's decentralization.

Snarkification refers to the process of using type of Zero Knowledge Proofs (ZKPs)
called SNARK (Succint Non-interactive Argument of Knowledge) in order to offload
a computation to one entity (one node, or a cluster that is a small subset of
the whole network), which performs the computation and generates a proof of
its validity. This proof can then be verified by the rest of the network
at fractional compute cost.



\section{ZkEVM}

ZkEVM, as mentioned in \ref{subsec:snarkifying_execution}, is a modified EVM,
which given state and list of transactions produces next state and a proof $\pi$
attesting to correctness of that execution. Primary use case of ZkEVMs today is
enabling "ZK rollups", L2s which execute a bundle of transactions offchain,
generate a proof and submit the bundle with the proof on the L1, where only the
proof is validated. Second use case, but not used today, is to enshrine ZkEVM
into EVM as described in \ref{subsec:snarkifying_execution}.

\subsection{Challenges of EVM}

EVM was not designed with ZK taken into consideration:

\begin{enumerate}
	\item Many EVM opcodes are complex and expensive to prove with ZK. This has
		led to different types of EVM compatibility among ZkEVMs \cite{VitalikButerinZkEVMTypes}.
	\item 256 bit word size means that ZK systems working over prime fields must
		include range proofs, which increases the ZkEVM complexity.
	\item It's harder to prove stack based VM. For example, Starkware developed
		Cairo \cite{Cairo}, a register based model in order to implement its ZkEVM.
		This requires a custom smart contract compiler.
	\item EVM storage uses Merkle Patricia Tries with Keccak, which are not
		ZK friendly and have huge proving overhead.
\end{enumerate}

Proving the entire EVM within a single circuit (one proof) is computationally
and economically unfeasible. Such a proof would be several megabytes in size,
making it too expensive to store on L1, let alone prove.

\subsection{Recursive proofs}

Recursive proofs combine benefits of ZKP systems with fast prover times (e.g.,
those based FRI) and systems with short proofs (like Groth16). The idea is to
produce a proof of a knowledge of a proof.

\begin{tikzFigure}
	\begin{tikzpicture}[
		diamondNode/.style={
			diamond, draw, thick,
			minimum width=4cm,
			minimum height=0.1cm,
			align=center, font=\sffamily
		},
		proverNode/.style={
			rectangle, rounded corners=10pt, draw, thick, fill=blue!20,
			minimum width=4cm,
			minimum height=1cm,
			text width=3.5cm,
			align=center, font=\sffamily
		},
		piNode/.style={
			circle, draw, thick, fill=white,
			minimum size=1.2cm,
			align=center, font=\sffamily\Large
		},
		piPrimeNode/.style={
			circle, draw, thick, fill=green!40,
			minimum size=1.2cm,
			align=center, font=\sffamily\Large
		},
		arrow/.style={
			-Stealth, thick, draw=black!80
		},
		labelOnArrow/.style={
			font=\sffamily\small, midway, sloped
		},
		annotationText/.style={
			text=red, font=\sffamily\small, align=center
		}
	]

	\node[proverNode] (inner) {Inner prover};
	\node[proverNode, right=2.5cm of inner] (outer) {Outer prover};

	\node[diamondNode, fill=red!25, above=0.8cm of inner] (witness) {witness};
	\node[diamondNode, fill=green!25, right=1cm of witness] (public) {public inputs};

	\node[piNode, below=1.2cm of inner.south] (pi) {$\pi$};
	\node[piPrimeNode, right=2cm of outer.east] (piprime) {$\pi'$};

	\draw[arrow] (witness.south) -- (inner.north);
	\draw[arrow] (public.east) -- (outer.north);
	\draw[arrow] (public.west) -- (inner.north);

	\draw[arrow] (inner.south) to node[font=\sffamily\small, below, xshift=-1cm] {generate} (pi.north);
	\draw[arrow] (pi.east) to[out=0, in=190, looseness=1] node[font=\sffamily\small, right] {Take in as witness} (outer.west);
	\draw[arrow] (outer.east) to node[font=\sffamily\small, below] {generate} (piprime.west);

	\end{tikzpicture}
	\caption{Recursive proving}
\end{tikzFigure}

The recursive ZKP is an instantiation of Incrementally-Verifiable-Computation (IVC)
\cite{ValiantIVC}. IVC is a cryptographic primitive for proving the
correctness of an iterated and incremental computation (such as EVM, or RISC-V
microprocessor). The output of step $i$ of the computation ($F$) is fed as input
into step $i+1$ of the computation, along with a proof $\pi$, which proves that:

\begin{enumerate}
	\item $F(s_{i-1}, \omega_i) = s_i$ - proves that executing $F$ with $s_{i-1}$
		and $\omega_i$ correctly outputs $s_i$,
	\item $V((i-1, s_0, s_{i-1}), \pi_{i-1}) = true$ - proves that $\pi_{i-1}$ is
		a valid proof relative to the previous step of the computation.
\end{enumerate}

\begin{tikzFigure}
	\begin{tikzpicture}[
		% settings for node distances
		node distance=1cm and 1.5cm, % vertical and horizontal base distance
		% Styles
		processNode/.style={ % For F1, F2, Fn
			rectangle, draw, fill=blue!20,
			minimum size=1cm, % makes F nodes squarish and compact
			align=center
		},
		dataNode/.style={ % For s0, sn, w1, w2, wn
			parallelogram, draw, fill=green!20,
			minimum height=0.5cm, % consistent height
			align=center,
			inner xsep=2pt, % padding inside the node
		},
		multiLineDataNode/.style={ % specific for s0 and sn with line breaks
			dataNode,
			text width=2.5cm % allows text to wrap
		},
		omegaDataNode/.style={ % specific for omega nodes
			dataNode,
			minimum width=1cm
		},
		ellipsisNode/.style={
			text centered,
			font=\small
		},
		arrow/.style={
			-Stealth,
			semithick
		},
		labelFont/.style={font=\footnotesize}
	]

	\node[multiLineDataNode] (s0) {$s_0$\\init state};
	\node[processNode, right=0.5cm of s0] (F1) {F};
	\node[processNode, right=of F1] (F2) {F};
	\node[ellipsisNode, right=1.5cm of F2] (dots) {...};
	\node[processNode, right=2cm of dots] (Fn) {F};
	\node[multiLineDataNode, right=0.5cm of Fn] (sn) {$(s_n, \pi_n)$\\final output};

	\node[omegaDataNode, below=of F1] (w1) {$\omega_1$};
	\node[omegaDataNode, below=of F2] (w2) {$\omega_2$};
	\node[omegaDataNode, below=of Fn] (wn) {$\omega_n$};

	\draw[arrow] (s0) -- (F1);
	\draw[arrow] (w1) -- (F1);
	\draw[arrow] (F1) -- node[midway, above, labelFont] {$(s_1, \pi_1)$} (F2);
	\draw[arrow] (w2) -- (F2);
	\draw[arrow] (F2) -- node[midway, above, labelFont] {$(s_2, \pi_2)$} (dots);
	\draw[arrow] (dots) -- node[midway, above, labelFont] {$(s_{n-1}, \pi_{n-1})$} (Fn);
	\draw[arrow] (wn) -- (Fn);
	\draw[arrow] (Fn) -- (sn);

	\end{tikzpicture}
	\caption{IVC \cite{ValiantIVC}}
\end{tikzFigure}

Final proof $\pi_n$ is a succinct proof that prover has $\omega_1, \ldots, \omega_n$,
s.t., final output $s_n$ is correct. The first implementation of IVC was done
using SNARKs \cite{RecursiveSNARKsIVC}

\subsection{First generation of ZkEVMs}

In the 2022 to early 2024 first generation of ZkEVMs was successfully deployed.
Thanks to recursive proving, it was feasible for these ZkEVMs to be created.
They first prove the EVM execution, and then prove it inside another circuit.
This enabled parallelization of proving and reduced the size and complexity of
the final proof submitted to L1. ZK systems like Plonky2 \cite{Plonky2}, or
Halo2 \cite{Halo2}, are examples of recursive ZKP systems. Notable references
include Scroll, zkSync Era, Polygon ZkEVM.

\subsection{Folding schemes}

Even though recursion proofs enabled teams to build ZkEVMs, they are not without
their flaws. Prover needs to have circuit which contains the whole verification
algorithm of another proof system. It must verify expensive evaluation proofs
for polynomial commitments.

Folding scheme takes almost all of the verification steps out of the circuit.
There are no FFTs, only multi-exponentiations, which do not require big
memory overhead. Also no embedded elliptic curve pairings are needed, because
there is no need to switch curves like in recursive proofs.

It compresses two instances into one. Folding prover will fold two instances,
with corresponding witnesses, and produce single instance, s.t., if it is correct,
it is implied that also the two original instances were correct \cite{Nova}.

Given some relations $R = \alpha(pp,\;x,\;W)$, where:
\begin{itemize}
	\item $pp$ = public parameters,
	\item $x$ = instance or public inputs,
	\item $W$ = witness.
\end{itemize}

For example: $R$ is an equation, $x$ are coefficients and $W$ are concrete
values that satisfy the equation.

A folding scheme for two relations $R$ and $R_{acc}$ (or $R_1$ and $R_2$) is an
interactive protocol between folding prover $P$ and folding verifier $V$ where:

\begin{enumerate}
	\item $P$ has $(pp, x_1, w_1) \in R, (pp, x_2, w_2) \in R_{acc}$,
	\item $V$ has $(pp, x_1), (pp, x_2)$.
\end{enumerate}

And the result of their interaction is $(pp, x_3, w_3)$ (where $V$ only knows
$(pp, x_3)$ and $w_3$ is kept private for $P$). Thus instead of validating two
instances, verifier only needs to check one. To make the interaction between
$P$ and $V$ non-interactive with \cite{FiatShamir}, $P$ must also provide
proof that Fiat-Shamir was used correctly. And also, $P$ needs to prove that
the folded instances are all linked together, output of step $i$ is input of
step $i+1$.

\begin{tikzFigure}
	\begin{tikzpicture}[
		node distance=1cm and 2cm, % vertical and horizontal base distance
		processNode/.style={
			rectangle, draw, fill=blue!20,
			minimum size=1cm,
			align=center
		},
		dataNode/.style={
			parallelogram, draw, fill=green!20,
			minimum height=0.5cm,
			align=center,
			inner xsep=2pt, % padding inside the node
		},
		omegaDataNode/.style={
			dataNode,
			minimum width=1cm
		},
		ellipsisNode/.style={
			text centered,
			font=\small
		},
		arrow/.style={
			-Stealth,
			semithick
		},
		labelFont/.style={font=\footnotesize}
	]

	\node[omegaDataNode] (w1) {$(\omega_1, x_1)$};
	\node[omegaDataNode, below=of w1] (w2) {$(\omega_2, x_2)$};
	\node[omegaDataNode, below=of F1, xshift=1cm] (w3) {$(\omega_4, x_4)$};
	\node[omegaDataNode, below=of dots, xshift=1cm, yshift=-.4cm] (wn1) {$(\omega_{n-1}, x_{n-1})$};
	\node[omegaDataNode, below=of Fn, xshift=3.1cm] (wn) {$(\omega_n, x_n)$};

	\node[processNode, right=0.5cm of w1] (F1) {Folding};
	\node[processNode, right=of F1] (F2) {Folding};
	\node[ellipsisNode, right=2cm of F2] (dots) {...};
	\node[processNode, right=3cm of dots] (Fn) {Folding};

	\draw[arrow] (w1) -- (F1);
	\draw[arrow] (w2) -- (F1);
	\draw[arrow] (F1) -- node[midway, above, labelFont] {$(w_3, x_3, \pi_3)$} (F2);
	\draw[arrow] (w3) -- (F2);
	\draw[arrow] (F2) -- node[midway, above, labelFont] {$(w_5, x_5, \pi_5)$} (dots);
	\draw[arrow] (dots) -- node[midway, above, labelFont] {$(w_{n-2}, x_{n-2}, \pi_{n-2})$} (Fn);
	\draw[arrow] (wn1) -- (Fn);
	\draw[arrow] (Fn) -- (wn);

	\end{tikzpicture}
	\caption{Folding scheme \cite{Nova}}
\end{tikzFigure}


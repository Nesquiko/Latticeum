\chapter{Related work}\label{chapter:related}

With the growth of cryptocurrencies, ZKPs have gained a substantial popularity
not only in academic circles, but also in the startup and venture capital
ones. Many researchers are studying this field and it is not considered 
as a cryptography for nerds anymore. 

On the other hand, there are not many contributions to the stealth address
ecosystem. This field gets some traction, for example, work done by
Fan, Jia and Wang, Zhen and Luo, Yili and Bai, Jian and Li, Yarong and Hao, Yao
\cite{FanJiaWang2019} introduce a stealth address scheme designed to
enhance user privacy in blockchain transactions. The scheme aims to address
the limitations of existing solutions, such as the need for users to manage
multiple key pairs or engage in additional private communication before each
transaction. In their approach, a user only needs to maintain a single key
pair for initial certification, simplifying key management and reducing
storage costs. The sender creates a one-time transaction address and attaches
it to the transaction, while the receiver utilizes their private key to verify
the transaction directly from the blockchain, eliminating the need for a
separate private channel. The scheme also offers flexibility for regulation,
allowing transactions to be either fully or partially regulated based on
security requirements.

Or, Wang \cite{Wang2023} introduced the concept of Fast Stealth Addresses (FSAs)
to improve the search efficiency of stealth address schemes in blockchain transactions.
Their approach aims to overcome the linear search time required in existing schemes
by allowing constant recognition time to determine if a block contains a recipient's
transactions and logarithmic search time to locate the specific transactions.
The authors provide a generic construction of the FSA scheme under subgroup membership
assumptions related to factoring and instantiate concrete schemes based on specific
number-theoretic assumptions. They also formalize the security model of an FSA
scheme and provide provable security analysis.

There are two protocols on Ethereum that implement stealth addresses.
First one is called Nocturne \cite{nocturne}. The protocol is a mix of
elliptic curve cryptography and ZKPs with few intermediate smart contracts.
In this protocol you lock your funds in the Nocturne ecosystem, in which
new stealth addresses are created with Elliptic curve stealth addresses scheme,
and you access your funds by submitting ZKPs. However according to their
\href{https://twitter.com/nocturne_xyz/status/1749510390906511693}{Twitter post}
they are discontinuing their protocol.

The second one is called Umbra \cite{umbra}. Umbra is a protocol on Ethereum
that implements stealth addresses using elliptic curve cryptography. In this
protocol, the recipient a private keys. The sender generates a random number,
encrypts it using the recipient's public key, and then
computes the stealth address from the recipient's public key and the random
number. The encrypted random number, ephemeral public key, and stealth address
are then published. The recipient can scan these announcements, decrypt the
random number using their private key, and check if the derived stealth
address matches the one in the announcement. If it does, they can then use
their spending key to access the funds.

Kovács and Seres \cite{Kovacs2023} conducted an analysis of Umbra's recipient
anonymity guarantees on Ethereum and its layer-2 solutions. They identified four
heuristics based on user behavior that could compromise the anonymity and
linkability of Umbra transactions. These heuristics exploit patterns such as
the reuse of registrant addresses, the overlap of sender and receiver
addresses, the collection of multiple payments to a single address, and unique
transaction fees. The study found that a significant portion of Umbra
transactions could be deanonymized using these heuristics, raising concerns
about the protocol's privacy in practice. The authors suggest countermeasures
such as avoiding address reuse and using different addresses for different
on-chain activities to mitigate these risks.


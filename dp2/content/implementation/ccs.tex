\section{Circuit synthesis}

The constraints defining the validity of the VM execution and the IVC chain
are synthesized into a CCS.

\subsection{Layout Management}

A central component of the synthesis is the \texttt{CCSLayout} structure. This
structure dynamically maps semantic variables (such as \texttt{is\_add},
\texttt{val\_rs1}, or specific Poseidon2 round constants) to column indices
within the CCS witness vector $z$.

This dynamic allocation allows for the easy addition of new constraints.
When a new variable is registered in the layout (e.g., adding support for a
new instruction), the layout automatically shifts all subsequent indices.
The witness generation logic in arithmetization and the constraint
definition logic in remain synchronized without manual recalculation of column
offsets.

\subsection{Poseidon2 constraint example}

This section illustrates the construction of CCS constraints for Poseidon2
hash. Poseidon2 permutation in Goldilocks field on 12 elements (rate is 12,
with capacity 4, total width then being 16) consists of:

\begin{enumerate}
	\item Applying a maximum distance separable MDS matrix.
	\item 4 external initial rounds a 7th degree S-box on all elements.
	\item 22 internal rounds with a partial S-box applied only to the first element.
	\item 4 external terminal rounds a 7th degree S-box on all elements.
\end{enumerate}

\paragraph{Initial MDS application.}

The first operation applies the MDS matrix to the absorbed state. The input 12 elements
and 4 zeroes must be in the witness. Here this input will be denoted as $s = (s_0, s_1, \ldots, s_{11}, 0, 0, 0, 0)$.
Then the output of the MDS application is captured into the witness, denoted as $s_{after\_mds}$.
The constraint enforces 

\[
	s_{after\_mds\_i} - s'_i = 0
\]

where $i \in (0, 1, \ldots, 15)$ and $s' = \texttt{MDS} \cdot s$.

\paragraph{External initial rounds.}

Next are external rounds where a round constant is added and S-box is applied
is applied to all state elements. There are 4 external initial rounds, after each
the state of 16 elements must be captured (denoted as
$s_{\text{ext\_init\_r\_i}}; \quad r \in (0, 1, 2, 3); \quad i \in (0, 1, \ldots, 15)$).
Then in circuit, for each round there is a constraint:

\begin{equation}
	\texttt{MDS}^{-1} \cdot s_{\text{ext\_init\_r\_i}} - (s_{\text{in\_r\_i}} + c_{\text{ext\_init\_r\_i}})^7 = 0
\end{equation}

where $c_{\text{ext\_init\_r\_i}}$ is the specific constant at index $i$ for round
$r$, and the $s_{\text{in\_r\_i}}$ is the input to the round, in first one, it
is the $s_{after\_mds}$, otherwise it is the output of previous round.

\paragraph{Internal rounds.}

Following the external initial rounds, the permutation executes 22 partial
internal rounds where the S-box is applied only to the first state element.
The state after each round is captured into witness, denoted as $s_{\text{inter\_r\_i}}; \quad r \in (0, 1, \ldots, 21); \quad i \in (0, 1, \ldots, 15)$.
Each internal round computes for each first element:

\begin{equation}
	M_I^{-1} * s_{\text{inter\_r\_0}} - (s_{\text{in\_r\_0}} + c_{\text{inter\_0}})^7 = 0
\end{equation}

For other than first element ($i \in (1, 2, \ldots, 15)$ it does:

\begin{equation}
	M_I^{-1} * s_{\text{inter\_r\_i}} - s_{\text{in\_r\_i}} = 0
\end{equation}

The $s_{\text{in\_0}}$ is the output of the last external initial round, other
$s_{\text{in\_i}}$ are outputs of the previous internal rounds.

\paragraph{External terminal rounds.}

The permutation concludes with 4 external terminal rounds, mirroring the
external initial rounds but using terminal round constants and different inputs
$s_{\text{ext\_term\_r\_i}}; \quad r \in (0, 1, 2, 3); \quad i \in (0, 1, \ldots, 15)$):

\begin{equation}
	\texttt{MDS}^{-1} \cdot s_{\text{ext\_term\_r\_i}} - (s_{\text{in\_r\_i}} + c_{\text{ext\_term\_r\_i}})^7 = 0
\end{equation}


where $c_{\text{ext\_term\_r\_i}}$ is the specific constant at index $i$ for round
$r$, and the $s_{\text{in\_r\_i}}$ is the input to the round, in first one, it
is the last internal round, otherwise it is the output of previous external terminal round.

\paragraph{Output extraction.}

After the final external terminal round, the output digest is extracted
from first 4 elements of the state

\begin{equation}
    h = (s_0, s_1, s_2, s_3)
\end{equation}

And compared to the claimed result.

\paragraph{Complexity analysis.}

For a single Poseidon2 sponge pass, the constraint system captures 388 witness
variables across the permutation computation. The system requires 44 sparse
matrices to encode the linear transformations, round constants, and S-box
operations. The number of matrices remains constant regardless of
the number of sponge passes, though their dimensions scale with the permutation
state size. The highest degree constraint is a 7th degree multiset constraint
corresponding to the Goldilocks S-box, and this degree does not increase with
additional sponge passes.

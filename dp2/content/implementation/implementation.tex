\chapter{Implementation}\label{chap:implementation}

This chapter presents the prototype implementation of the lattice-based RISC-V
ZkVM architecture described in the Chapter~\ref{chap:design}. As of writing this work,
the implementation covers a subset of RISC-V instructions sufficient to run
simple programs such as an Nth Fibonacci number computation, the CCS
constraints for these instructions, and the CCS constraints for the Poseidon2
IVC step commitment. The circuit NIFS verifier and the final SuperSpartan+Greyhound wrapper
are under development.

Regarding the choice of folding scheme, this work utilizes the LatticeFold
protocol rather than its successor LatticeFold+, as the latter's library
implementation was not available at the time of system design and development.
In future migration to LatticeFold+ or other folding schemes
is possible.

\section{Technology stack}\label{sec:tech_stack}

The prototype implementation of the lattice-based RISC-V ZkVM is developed
using the Rust programming language (\texttt{nightly-2025-08-19}). Rust was
selected due to its dominance in the cryptography and zero-knowledge ecosystem.

The cryptographic foundations and virtual machine components rely on several
crates:

\begin{itemize}
    \item \textbf{Plonky3\footnote{\url{https://github.com/Plonky3/Plonky3}}:} The \texttt{Plonky3} meta crate is utilized to
		provide the core arithmetization primitives. The \texttt{p3-field} and
		\texttt{p3-goldilocks} crates are used for Goldilocks field
		arithmetic. \texttt{p3-poseidon2} for the construction of Poseidon2
		hashes and \texttt{p3-merkle-tree} for memory and code commitments.
    
    \item \textbf{LatticeFold\footnote{\url{https://github.com/NethermindEth/latticefold}}:} The core folding mechanism is based on the
		\texttt{latticefold} implementation provided by Nethermind. This
		crate implements the Ajtai commitment scheme and the NIFS prover and
		verifier logic. It is the primary component of the
		IVC loop, enabling the recursive accumulation of execution steps.
    
    \item \textbf{Cyclotomic-rings\footnote{\url{https://github.com/NethermindEth/latticefold}}:} From the same source as LatticeFold crate,
		the \texttt{cyclotomic-rings} crate is used to handle the specific
		algebraic structures required for lattice-based cryptography over ring
		$R_q = \mathbb{Z}_q[X]/(X^d+1)$. It provides the necessary NTT
		implementations and ring arithmetic for polynomials over the
		Goldilocks field.
\end{itemize}


\section{Cryptography}

Cryptographic crates (and majority of all cryptographic libraries, not just for Rust)
expose high level API. Developers provide inputs to this API and get back the
desired hash, encrypted text, key, etc. In context of ZkVM, also the intermediate
states are needed to populate the witness. Main hurdle during implementation
was to reuse exposed parts of used cryptographic crates, in order to collect
internal states of these primitives.

\subsection{Poseidon2}

The Poseidon2 hash function is used to commit to the IVC state. While the
\texttt{Plonky3} library provides a implementation of Poseidon2, its standard
interface treats the permutation as a black box, returning only the final digest.
To correctly constrain the Poseidon2 execution within a CCS requires access to
the intermediate states of the permutation. This is because Poseidon2 has
a 7th degree S-box, which is dependent on previous rounds, and those on ones
before them. Thus, the computation is broken down into smaller
steps, where each step verifies a transition of a one computational segment.

A custom implementation of the Poseidon2 permutation was
constructed, structurally identical to the \texttt{Plonky3} implementation but
modified to capture and return the state vector at desired boundaries:

\begin{itemize}
    \item The state after the initial MDS matrix application.
    \item The state after each of the external initial full rounds.
    \item The state after each of the internal partial rounds.
    \item The state after each of the external terminal full rounds.
\end{itemize}

\subsection{LatticeFold Integration}

The LatticeFold library is utilized to perform the non-interactive folding of
the witness-instance pairs. The internal logic of the LatticeFold prover was
not modified to expose intermediate states, because this step was not implemented
as of writing this work.


\section{RISC-V Emulator}

A custom, lightweight RISC-V interpreter was developed to execute the program
and generate the necessary cryptographic witnesses. The emulator supports the
\texttt{rv32imac} instruction set architecture. Since it being only a PoC, not
all instruction handlers are implemented, but sufficient amount of them to execute
Rust binaries compiled for the \texttt{riscv32imac-unknown-none-elf} target.

The core of the emulator is structured around an execution loop that fetches,
decodes, and executes instructions. At each instruction step, the execution
halts and the full machine state, including the program
counter, registers, memory pages, and raw instruction bytes, is exposed to an
interceptor.

This way, the instruction semantics and the proof generation are decoupled. The
emulator is responsible solely for correct state transitions and the
generation of the execution trace, while the ZkVM utilizes this trace to
generate witnesses.


\section{Circuit synthesis}

The constraints defining the validity of the VM execution and the IVC chain
are synthesized into a CCS.

\subsection{Layout Management}

A central component of the synthesis is the \texttt{CCSLayout} structure. This
structure dynamically maps semantic variables (such as \texttt{is\_add},
\texttt{val\_rs1}, or specific Poseidon2 round constants) to column indices
within the CCS witness vector $z$.

This dynamic allocation allows for the easy addition of new constraints.
When a new variable is registered in the layout (e.g., adding support for a
new instruction), the layout automatically shifts all subsequent indices.
The witness generation logic in arithmetization and the constraint
definition logic in remain synchronized without manual recalculation of column
offsets.

\subsection{Poseidon2 constraint example}

This section illustrates the construction of CCS constraints for Poseidon2
hash. Poseidon2 permutation in Goldilocks field on 12 elements (rate is 12,
with capacity 4, total width then being 16) consists of:

\begin{enumerate}
	\item Applying a maximum distance separable MDS matrix.
	\item 4 external initial rounds a 7th degree S-box on all elements.
	\item 22 internal rounds with a partial S-box applied only to the first element.
	\item 4 external terminal rounds a 7th degree S-box on all elements.
\end{enumerate}

\paragraph{Initial MDS application.}

The first operation applies the MDS matrix to the absorbed state. The input 12 elements
and 4 zeroes must be in the witness. Here this input will be denoted as $s = (s_0, s_1, \ldots, s_{11}, 0, 0, 0, 0)$.
Then the output of the MDS application is captured into the witness, denoted as $s_{after\_mds}$.
The constraint enforces 

\[
	s_{after\_mds\_i} - s'_i = 0
\]

where $i \in (0, 1, \ldots, 15)$ and $s' = \texttt{MDS} \cdot s$.

\paragraph{External initial rounds.}

Next are external rounds where a round constant is added and S-box is applied
is applied to all state elements. There are 4 external initial rounds, after each
the state of 16 elements must be captured (denoted as
$s_{\text{ext\_init\_r\_i}}; \quad r \in (0, 1, 2, 3); \quad i \in (0, 1, \ldots, 15)$).
Then in circuit, for each round there is a constraint:

\begin{equation}
	\texttt{MDS}^{-1} \cdot s_{\text{ext\_init\_r\_i}} - (s_{\text{in\_r\_i}} + c_{\text{ext\_init\_r\_i}})^7 = 0
\end{equation}

where $c_{\text{ext\_init\_r\_i}}$ is the specific constant at index $i$ for round
$r$, and the $s_{\text{in\_r\_i}}$ is the input to the round, in first one, it
is the $s_{after\_mds}$, otherwise it is the output of previous round.

\paragraph{Internal rounds.}

Following the external initial rounds, the permutation executes 22 partial
internal rounds where the S-box is applied only to the first state element.
The state after each round is captured into witness, denoted as $s_{\text{inter\_r\_i}}; \quad r \in (0, 1, \ldots, 21); \quad i \in (0, 1, \ldots, 15)$.
Each internal round computes for each first element:

\begin{equation}
	M_I^{-1} * s_{\text{inter\_r\_0}} - (s_{\text{in\_r\_0}} + c_{\text{inter\_0}})^7 = 0
\end{equation}

For other than first element ($i \in (1, 2, \ldots, 15)$ it does:

\begin{equation}
	M_I^{-1} * s_{\text{inter\_r\_i}} - s_{\text{in\_r\_i}} = 0
\end{equation}

The $s_{\text{in\_0}}$ is the output of the last external initial round, other
$s_{\text{in\_i}}$ are outputs of the previous internal rounds.

\paragraph{External terminal rounds.}

The permutation concludes with 4 external terminal rounds, mirroring the
external initial rounds but using terminal round constants and different inputs
$s_{\text{ext\_term\_r\_i}}; \quad r \in (0, 1, 2, 3); \quad i \in (0, 1, \ldots, 15)$):

\begin{equation}
	\texttt{MDS}^{-1} \cdot s_{\text{ext\_term\_r\_i}} - (s_{\text{in\_r\_i}} + c_{\text{ext\_term\_r\_i}})^7 = 0
\end{equation}


where $c_{\text{ext\_term\_r\_i}}$ is the specific constant at index $i$ for round
$r$, and the $s_{\text{in\_r\_i}}$ is the input to the round, in first one, it
is the last internal round, otherwise it is the output of previous external terminal round.

\paragraph{Output extraction.}

After the final external terminal round, the output digest is extracted
from first 4 elements of the state

\begin{equation}
    h = (s_0, s_1, s_2, s_3)
\end{equation}

And compared to the claimed result.

\paragraph{Complexity analysis.}

For a single Poseidon2 sponge pass, the constraint system captures 388 witness
variables across the permutation computation. The system requires 44 sparse
matrices to encode the linear transformations, round constants, and S-box
operations. The number of matrices remains constant regardless of
the number of sponge passes, though their dimensions scale with the permutation
state size. The highest degree constraint is a 7th degree multiset constraint
corresponding to the Goldilocks S-box, and this degree does not increase with
additional sponge passes.


\section{High level overview}

Figure~\ref{fig:execution_flow_overview} illustrates the complete execution flow. The
setup phase prepares the VM by loading the ELF binary and constructing the CCS
constraint system. Within the execution loop, four main modules work together:
the \texttt{riscvm} module handles instruction execution and trace generation,
the \texttt{commitments} module creates Poseidon2 hashes for state components,
the \texttt{ivc} module arithmetizes the trace into the CCS witness vector,
and the \texttt{main} module performs the folding operations that
compress each step into the accumulator. The dashed line on the right shows the
recursive nature of the computation, where the \texttt{IVCStepOutput} from step
$i$ becomes the \texttt{IVCStepInput} for step $i+1$.

\begin{tikzFigure}
	\begin{tikzpicture}[
			node distance=1.8cm and 1.0cm,
			font=\sffamily\small,
			% --- Styles ---
			% Functions / Logic
			func/.style={
				rectangle, 
				draw=red!80!black, 
				fill=red!10, 
				thick, 
				minimum height=0.7cm, 
				minimum width=4.5cm,
				align=center,
				rounded corners=2pt
			},
			% Data Structures
			data/.style={
				rectangle, 
				draw=blue!80!black, 
				fill=blue!5, 
				thick, 
				minimum height=0.6cm, 
				minimum width=4cm,
				align=center,
				dashed,
				font=\sffamily\footnotesize
			},
			% Module Containers
			module/.style={
				draw=gray!40, 
				dashed, 
				inner sep=0.5cm, 
				rounded corners=5pt,
				fill=white,
				fill opacity=0.0
			},
			% Module Label
			modlabel/.style={
				font=\bfseries\scriptsize,
				text=gray!60!black,
				anchor=north east
			},
			% Lines
			line/.style={
				draw, 
				-Latex, 
				thick, 
				gray!70!black
			}
			]

			% --- 1. SETUP PHASE (Top) ---
			\node[func] (setup_vm) {\texttt{VM::load\_elf}};
			\node[func, right=0.5cm of setup_vm] (setup_ccs) {\texttt{CCSBuilder::create\_riscv\_ccs}};

			% --- 2. THE LOOP  ---

			% --- Module: riscvm ---
			\node[func, below=1.8cm of setup_vm, xshift=2.5cm] (vm_exec) {\texttt{VM::fetch\_execute}};
			\node[data, below=0.3cm of vm_exec] (trace) {\texttt{ExecutionTrace}};

			% --- Module: commitments ---
			\node[func, below=1.5cm of trace] (committer) {
				\textbf{ZkVmCommitter methods:} \\
				ivc\_step\_comm,\\
				state\_i\_comm,\\
				acc\_comm,\\
				vm\_regs\_comm,\\
				vm\_mem\_comm
			};
			\node[data, below=0.3cm of committer] (comms) {Commitments represented by\\\texttt{GoldilocksComm}\\(4 Goldilocks field elements)};

			% --- Module: ivc ---
			\node[data, below=1.5cm of comms] (ivc_input) {\textbf{\texttt{IVCStepInput}} \\ (Prev Acc + Trace + Comms)};
			\node[func, below=0.3cm of ivc_input] (arith) {\texttt{arithmetize(input, layout)}};
			\node[data, below=0.3cm of arith] (z_vec) {\texttt{Vec<GoldilocksRingNTT>}\\($z$ vector)};

			% --- Module: main ---
			\node[func, below=1.5cm of z_vec] (commit_wit) {\texttt{commit(z, ccs, scheme)}};
			\node[data, below=0.3cm of commit_wit] (cccs) {\texttt{CCCS} ($cm_i$) + \texttt{Witness} ($w_i$)};

			\node[func, below=0.3cm of cccs] (nifs) {\texttt{fold(ccs, scheme, acc, w\_acc, cm\_i, w\_i)}};
			\node[data, below=0.3cm of nifs] (output) {\textbf{\texttt{IVCStepOutput}} \\ (New Accumulator + Proof)};

			% --- 3. MODULE BOUNDARIES ---
			\begin{scope}[on background layer]
				% VM Box
				\node[module, fit=(vm_exec) (trace)] (box_riscv) {};
				\node[modlabel] at (box_riscv.north east) {\texttt{mod riscvm}};

				% Comm Box
				\node[module, fit=(committer) (comms)] (box_comm) {};
				\node[modlabel] at (box_comm.north east) {\texttt{mod commitments}};

				% IVC Box
				\node[module, fit=(ivc_input) (arith) (z_vec)] (box_ivc) {};
				\node[modlabel] at (box_ivc.north east) {\texttt{mod ivc}};

				% Latticefold Box
				\node[module, fit=(commit_wit) (cccs) (nifs) (output)] (box_fold) {};
				\node[modlabel] at (box_fold.north east) {\texttt{crate latticefold}};

				% Main Loop Container
				\node[draw=gray!80, thick, inner sep=25pt, fit=(box_riscv) (box_fold), label={[anchor=north west, font=\bfseries]north west:Execution Loop (\texttt{vm.run})}] (main_loop) {};
			\end{scope}

			% --- 4. CONNECTIONS ---

			% Init -> Loop
			% Manual orthogonal lines
			\draw[line] (setup_vm.south east) -- (vm_exec.north);
			\draw[line] (setup_ccs.south east) -- ++(0,-4.5) -- (commit_wit.east);

			% Flow down
			\draw[line] (vm_exec) -- (trace);
			\draw[line] (trace) -- (committer);
			\draw[line] (committer) -- (comms);
			\draw[line] (comms) -- (ivc_input);
			\draw[line] (ivc_input) -- (arith);
			\draw[line] (arith) -- (z_vec);
			\draw[line] (z_vec) -- (commit_wit);
			\draw[line] (commit_wit) -- (cccs);
			\draw[line] (cccs) -- (nifs);
			\draw[line] (nifs) -- (output);

			\draw[line, dashed] (output.east) -- ++(2.9, 0) 
			-- ++(0, 7.3)
			-- (ivc_input.east);

			% Label for loop
			\node[font=\scriptsize\bfseries, text=gray!80!black, rotate=90] at ($(output.east) + (2.7, 3.5)$) {Accumulator ($i \to i+1$)};
		\end{tikzpicture}
   \caption{High-level overview of the implemented execution flow.}
    \label{fig:execution_flow_overview}
\end{tikzFigure}


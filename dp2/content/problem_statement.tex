\chapter{Problem statement and research objectives}\label{chap:problem_statement}

Based on the analysis of the current state of ZkVMs and the theoretical
properties of lattice-based cryptography, this chapter defines the specific
problems addressed by this thesis. It outlines the research gap regarding
post-quantum secure ZkVMs and establishes the specific scientific questions
and objectives that guide the subsequent solution design and implementation.

\section{Problem statement}

The integration of ZKPs into blockchain architectures, particularly Ethereum,
offers a path toward massive scalability. As detailed in the previous chapter,
the industry standard has converged on ZkVMs that execute instruction sets
like RISC-V. These systems allow developers to write programs in more familiar
languages (such as Rust or C++) and prove their correct execution.

However, the majority of currently deployed ZkEVMs and ZkVMs rely on
cryptographic hardness assumptions, specifically the Discrete Logarithm Problem
in elliptic curve groups, that are vulnerable to attacks by sufficiently
powerful quantum computers. While hash-based STARKs offer a post-quantum
alternative, they often incur significant costs in terms of proof size or
verification complexity compared to SNARKs based on specific algebraic structures.

Lattice-based cryptography presents a promising third path. It offers
post-quantum security based on the hardness of the Shortest Vector Problem
and the Module Short Integer Solution \cite{AjtaiLattices, LatticeTutorial}.
Furthermore, lattice-based commitment schemes possess additive homomorphic
properties that make them theoretically suitable for highly efficient folding
schemes, such as LatticeFold \cite{LatticeFold}.

Despite these theoretical advantages, a significant gap exists in the
literature and open-source ecosystem, \textbf{there is currently no practical
implementation of a general-purpose RISC-V ZkVM built entirely upon lattice-based primitives.}

The problem addressed by this thesis is the lack of a concrete instantiation
of a RISC-V ZkVM that leverages lattice-based folding schemes. It remains
unproven whether the theoretical efficiency of lattice folding translates to
practical performance for complex computations like a VM. Without such an
implementation, it is impossible to accurately compare the performance
trade-offs between lattice-based approaches and the elliptic curve architectures.

\section{Research objectives}

The primary objective of this work is to design and implement a functional
prototype of a RISC-V ZkVM using lattice-based cryptography, specifically
leveraging the LatticeFold protocol for instantiating Incrementally Verifiable
Computation of a RISC-V VM. This implementation aims to serve as a
proof-of-concept for post-quantum execution environment for proving
Ethereum blocks. The specific objectives are defined as follows:

\begin{itemize}
    \item \textbf{Architectural Design: } Design a system architecture for a
		ZkVM that maps the RISC-V instruction set architecture to the
		constraints of the LatticeFold folding scheme. This involves defining
		the arithmetization of 32-bit CPU logic into a Customizable Constraint
		Systems (CCS) \cite{CCS} structure, suitable to be used by LatticeFold.
    \item \textbf{Implementation:} Implement the proposed design using the
		Rust programming language, integrating with the available LatticeFold
		libraries.
    \item \textbf{Comparative evaluation:} Evaluate the performance of the
		implemented lattice-based ZkVM. The goal is to benchmark the system
		against establishing metrics for proving time and memory usage,
		comparing these results with theoretical expectations and, where
		applicable, existing non-post-quantum baselines.
\end{itemize}

\section{Scientific Questions}

To achieve the stated objectives, this thesis seeks to answer the following
scientific questions:

\begin{enumerate}
    \item \textbf{Feasibility of lattice-based instantiation:} \textit{Is it
		feasible to instantiate a full RISC-V execution environment using
		CCS compatible with the LatticeFold protocol?}
    \item \textbf{Performance trade-offs:} \textit{How does the performance of
		a lattice-based folding scheme compare to traditional
		elliptic-curve-based folding schemes when applied to general-purpose
		VM execution?}
    \item \textbf{Suitability for long-running computations:} \textit{Does the
		noise growth inherent in lattice-based cryptography hinder the ability
		to prove long execution traces, such as those required for Ethereum block validation?}
\end{enumerate}


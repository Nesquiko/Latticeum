% Annotation in Slovak
\thispagestyle{empty}

\vspace*{\fill}

\section*{Anotácia}

\begin{minipage}[t]{1\columnwidth}
    \FIITuniversitySK

    \FIITfacultySK

    Študijný program: \FIITstudyProgramSK\\

    Autor: \FIITauthor

    \FIITthesisSK: \FIITtitleSK

    Vedúci bakalárskej práce: \FIITsupervisor

    \FIITdateSK
\end{minipage}

\bigskip{}

Táto práca skúma návrh a potenciálnu implementáciu riešení využívajúcich
dôkazy s nulovým vedomím (ZKP) v rámci ekosystému Ethereum, s cieľom riešiť
problémy škálovateľnosti a dosiahnuť post-kvantovú bezpečnosť. Dôkazy s
nulovým vedomím sú kryptografické metódy, ktoré umožňujú overenie integrity
výpočtov bez odhalenia citlivých dát, čo ich robí mimoriadne vhodnými pre
zlepšenie súkromia a efektivity v blockchainových systémoch. Ústredným
zameraním tejto práce je prieskum Zero-Knowledge Virtual Machine (ZkVM)
prispôsobenej pre inštrukčnú sadu RISC-V, s využitím princípov kryptografie
založenej na mriežkach. Kľúčovou aplikáciou takéhoto ZkVM by bolo dokázateľne
spávne vykonanie Ethereum blokov cez ZkEVM.

Výskum zahŕňa analýzu architektúry Ethereum, konceptu "snarkifikácie" pre
konsenzuálnu aj exekučnú vrstvu, evolúciu a prirodzené výzvy ZkEVM (vrátane
rekurzívnych dôkazov a skladacích schém), a preskúmanie kryptografických
primitív založených na mriežkach, ako sú Ajtaiho záväzky, spolu so súčasnými
systémami LaBRADOR, Greyhound a LatticeFold/LatticeFold+.

Cieľom tejto diplomovej práce je prispieť do odboru zhodnotením potenciálnej
výkonnosti ZKP systémov postavených na kryptografii založenej na mriežkach pre
úlohu dokazovania správneho vykonania Ethereum blokov. Práca sa ďalej snaží
porovnať tieto nové riešenia s etablovanými klasickými ZKP prístupmi, pričom
zachováva post-kvantovú odolnosť.


\newpage{}\thispagestyle{empty}\medskip{}

% Annotation in English
\thispagestyle{empty}

\vspace*{\fill}

\section*{Annotation}

\begin{minipage}[t]{1\columnwidth}
    \FIITuniversity

    \FIITfaculty

    Degree Course: \FIITstudyProgram\\

    Author: \FIITauthor

    \FIITthesis: \FIITtitle

    Supervisor: \FIITsupervisor

    \FIITdate
\end{minipage}

\bigskip{}

This thesis explores the design and potential implementation of
solutions utilizing Zero-Knowledge Proofs (ZKPs) within the Ethereum
ecosystem, aiming to address scalability challenges and achieve post-quantum
security. ZKPs are cryptographic methods that allow the verification of
computational integrity without revealing sensitive data, making them
suitable for improving privacy and efficiency in blockchain systems. The
central focus of this work is the exploration of a Zero-Knowledge Virtual
Machine (ZkVM) for the RISC-V instruction set, leveraging lattice-based
cryptography. A key application of such a ZkVM would be to enable creating
proofs of correct execution of EVM.

The research includes an analysis of Ethereum's architecture, the concept of
"snarkification" for both consensus and execution layers, the evolution and
challenges of ZkEVMs (recursive proofs and folding schemes), and an
examination of lattice-based cryptographic primitives, such as Ajtai
commitments, along with contemporary systems like LaBRADOR, Greyhound, and
LatticeFold/LatticeFold+.

This engineer's thesis aims to contribute to the field by evaluating the
potential performance of ZKP systems built on lattice-based cryptography for
the task of proving the correct execution of Ethereum blocks. The work further
seeks to compare these emerging solutions with established classical ZKP
approaches, while maintaining post-quantum resistance.

\newpage{}\thispagestyle{empty}

\emptypage

\chapter{Conclusion}\label{chap:conclusion}

This thesis presented the design and initial implementation of a post-quantum
ZkVM based on lattice-based cryptography. The proposed architecture integrates a
RISC-V execution environment with the LatticeFold folding scheme, supported by
CCS for arithmetization. The design provides
a theoretical blueprint, addressing the challenges of post-quantum
security and incrementally verifiable computation.

The implementation validates core component integration. The RISC-V emulator
successfully generates execution traces, CCS constraints for Poseidon2 hashes are
synthesized and integrated with LatticeFold, and the IVC step commitment
mechanism is functional. The performance baseline of approximately 42 seconds
for a 16 cycles long Fibonacci computation provides a positive outlook.

\section{Future Work}

Several directions for future research and development emerge from this work:

\begin{itemize}
    \item \textbf{Formal verification:} Formal verification of the implementation
		would provide strong guarantees of correctness.
    
    \item \textbf{LatticeFold+ integration:} Migrating the folding backend to
		LatticeFold+ could yield significant performance improvements through its
		more efficient algebraic range proofs and double commitment techniques.
    
    \item \textbf{GPU acceleration:} The LatticeFold prover operations, including
		NTT transformations and matrix-vector multiplications can be sped up with
		GPU acceleration, thus bringing proving times down.
    
	\item \textbf{Lookup incorporation:} Integrating lookup arguments could
		optimize constraint generation for certain operations, potentially reducing
		circuit size and improving performance for complex instructions.
\end{itemize}

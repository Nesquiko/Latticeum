\section{Technology stack}\label{sec:tech_stack}

The prototype implementation of the lattice-based RISC-V ZkVM is developed
using the Rust programming language (\texttt{nightly-2025-08-19}). Rust was
selected due to its dominance in the cryptography and zero-knowledge ecosystem.

The cryptographic foundations and virtual machine components rely on several
crates:

\begin{itemize}
    \item \textbf{Plonky3\footnote{\url{https://github.com/Plonky3/Plonky3}}:} The \texttt{Plonky3} meta crate is utilized to
		provide the core arithmetization primitives. The \texttt{p3-field} and
		\texttt{p3-goldilocks} crates are used for Goldilocks field
		arithmetic. \texttt{p3-poseidon2} for the construction of Poseidon2
		hashes and \texttt{p3-merkle-tree} for memory and code commitments.
    
    \item \textbf{LatticeFold\footnote{\url{https://github.com/NethermindEth/latticefold}}:} The core folding mechanism is based on the
		\texttt{latticefold} implementation provided by Nethermind. This
		crate implements the Ajtai commitment scheme and the NIFS prover and
		verifier logic. It is the primary component of the
		IVC loop, enabling the recursive accumulation of execution steps.
    
    \item \textbf{Cyclotomic-rings\footnote{\url{https://github.com/NethermindEth/latticefold}}:} From the same source as LatticeFold crate,
		the \texttt{cyclotomic-rings} crate is used to handle the specific
		algebraic structures required for lattice-based cryptography over ring
		$R_q = \mathbb{Z}_q[X]/(X^d+1)$. It provides the necessary NTT
		implementations and ring arithmetic for polynomials over the
		Goldilocks field.
\end{itemize}

\chapter{Analysis}\label{analysis}

This chapter, analyzes shift towards smaller prime fields in
ZKP systems to enhance efficiency and data density.
Section~\ref{analysis:smaller-prime-fields} discusses the shift from large
fields to smaller ones, highlighting protocols like Plonky2, STWO, and Plonky3
that leverage smaller primes for improved performance. Section~\ref{analysis:binary-fields}
delves into the use of binary field ($\mathbb{F}_2$), exploring their
computational advantages and the protocol proposed by Diamond and Posen
\cite{Binius} that utilizes towers of binary fields. Aim of this chapter is to
understand the potential benefits and challenges of adopting smaller and
binary fields in ZKP proving systems.

\section{Smaller prime fields}\label{analysis:smaller-prime-fields}

Today's ZK proving systems work over a large primary fields of bit size $2^{256}$.
However, the majority of programs use small numbers. Indices of arrays,
variables with 64 bit size, or values representing single bit (true or false)
use only a fraction of the whole $2^{256}$ field, thus creating an inefficiency
and decreasing the information density.

Current trend and research directions tend towards using smaller prime fields.
SNARKs over a elliptic curves become insecure when smaller prime fields are used.
On the other hand, STARKs \cite{SassonSTARKs} use different approach based on hashing.
This make it possible to reduce the size of the field. Plonky2 \cite{Plonky2}
started this by performing calculation over a $2^{64}$, which improved the proof
generation performance. Starkware's stwo \cite{CircleStarks} and Plonky3 \cite{Plonky3}
shrink the underlying field size further with usage of Mersenne prime $2^{31} - 1$.

\section{Binary field}\label{analysis:binary-fields}

This tendency to shrink underlying field has a logical conclusion, a field over
the smallest prime, 2. This field has a beautiful properties when computation is
done in it. Addition is a bitwise XOR without the need to carry. Squaring elements
is less expensive than multiplying two elements, due to the fact that in this
field $(x + y)^2 = x^2 + y^2$ (this property can be referred to as "Freshman's
dream \cite{FreshmansDream}).

Diamond and Posen in \cite{Binius} propose a protocol constructed from binary
field and binary tower of fields ($F_2 \subset F_{2^2} \subset F_{2^3}
\ldots$). The binary field can be extended as many 
times as needed \cite{Wiedemann86}. By using the binary field, data of size
$n$ will be encoded in $n$ bits, and hence creating a dense encoding.

Multilinear polynomial is committed with a Merkle tree. In order to
encode a polynomial representing large set of values, they need to be
accessible as evaluations of the polynomial and used field must contain such
values. So, the values (the trace of the computation) are encoded as points on
hypercube $P(x_0, x_1, \ldots, x_k)$. Then to prove evaluations at random
points, the data is interpreted as a square, extended with Reed-Solomon
encoding. This gives the data redundancy for random Merkle tree queries, so
that the evaluation is secure. And thanks to the binary field, the integers
produced by extending with Reed-Solomon do not blow up.

The proposed protocol has a $\mathcal{O}(\sqrt{N})$ verification time and
for a proof of $2^{32}$ bits around 11MB is needed.


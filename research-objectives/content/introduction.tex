\chapter{Introduction}

Zero Knowledge Proofs (ZKPs) are a cryptographical primitive, with which one
party (the prover) can prove to another party (the verifier) that a given
statement is true, without revealing any additional information beyond the
validity of the statement itself. This property is
used for maintaining privacy and security in various digital interactions,
where sensitive information must be protected, or a proof of a valid computation
is wanted. \cite{Goldreich1991}

The foundation of what later became ZKPs, were introduced in the seminal work
by Goldwasser, Micali, and Rackoff in 1985 \cite{Goldwasser1989}.
They developed a computational complexity theory focusing on what does it mean
to know, or possess some information. They described a different kind of proof,
an interactive proof. This prove involves two parties (prover and verifier),
which create a dialogue proving a truth of a statement. \cite{Goldwasser1986}

Main limitation of these proofs is that they require a constant communication
between the prover and verifier. This limitation was addressed by Fiat and
Shamir in 1986, who proposed a  method to transform interactive proofs into
non interactive ones using a random oracle \cite{Fiat1986}. This transformation
removed the requirement for the prover and verifier to be simultaneously online,
and hence there's no need for an interaction.

Nowadays, ZKPs' main application is scaling blockchain systems, like Ethereum
\cite{Ethereum}. ZK Rollups are layer 2 scaling solutions that leverage ZKPs
to increase the throughput of blockchain while maintaining security
and decentralization. They work by bundling multiple transactions offchain
and generating a proof that verifies the validity of these transactions,
which is verified on Ethereum. Another examples of ZK application are Coin
mixers, for instance TornadoCash, or decentralized identity.


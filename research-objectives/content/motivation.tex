\chapter{Motivation}

Despite their powerful capabilities, modern ZKP systems often face significant
practical challenges. Proving systems today typically require high performance
computing resources, sometimes even customized ASICs, to achieve reasonable
proving times. This reliance on non-consumer-grage hardware not only raises
the cost barrier for participation but also undermines the decentralization
ethos central to many cryptographic applications.

The motivation behind this work is to explore avenues for enhancing ZKPs by
reducing their hardware requirements. By slightly decreasing the computational
demands of proof generation, this work aims to enhance ZKPs and make them accessible to
individuals using common hardware, akin to how Ethereum staking allows
validators to participate without specialized equipment. Lowering the entry
barrier will support greater decentralization, enabling a wider community to
engage in proof generation and verification processes.

In pursuit of this goal, the potential of utilizing smaller prime fields and
binary fields in ZKP systems is investigated. By optimizing the underlying
mathematical structures, we hope to enhance efficiency and data density,
thereby reducing the need for specialized hardware. This approach could lead
the way for more scalable and decentralized cryptographic solutions,
benefiting a broad spectrum of applications and users.

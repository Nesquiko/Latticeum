\chapter{Research Objectives}

The primary objective of this work is to explore potential ways for
enhancing ZKP systems by addressing their current limitations, particularly
the high hardware requirements for efficient proof generation. This
exploration is guided by the following research questions:

\begin{enumerate}
	\item \textbf{Is there room for improving today's proving systems?}\\
		Analyzing whether existing ZKP systems have untapped potential
		for optimization, especially in terms of computational efficiency and
		resource utilization.

	\item \textbf{Is the minification of underlying mathematical structures a
	way to achieve this improvement?}\\ Examining the possibility that
	reducing the size of the finite fields or employing simpler mathematical
	structures can lead to more efficient ZKP systems without compromising security.

	\item \textbf{Are the underlying principles correct, and is the only way
	to enhance these systems to find new optimizations of the operations they
	use?}\\ Assessing whether the foundational principles of current ZKP
	systems are sound and determining if enhancements are achievable through
	optimizing existing operations rather than altering the fundamental
	mathematical frameworks.

	\item \textbf{Perhaps the limit has been reached, and only stronger
	hardware is the answer?}\\ Considering the possibility that current ZKP
	systems are already optimized to their theoretical limits, and any further
	improvements in proving times and efficiency would require advancements in
	hardware capabilities rather than software or algorithmic enhancements.
\end{enumerate}

This work aims to address these questions by analyzing the shift towards
smaller prime fields and the utilization of binary fields in ZKP systems, as
discussed in Chapter~\ref{analysis}. By evaluating the potential
benefits and challenges associated with minifying the underlying mathematical
structures, the study tries to determine whether such approaches can
reduce hardware requirements and improve the accessibility and
decentralization of ZKP technologies.
